%!TEX root =  ../samplepaper.tex
\section{Requirements}
Before we introduce the proposed algorithm, we must clarify its main goal. We want our algorithm to be able to support advanced feedback for students learning and practicing ND proofs. Such effective feedback system should be able to deliver relevant information to assist students at any stage of their exercise resolution. Focusing on this main goal, we identified four fundamental aspects a well-designed feedback system should satisfy:

\begin{itemize}

\item \textbf {Providing guidance on rule applications:} Some rule applications in ND are not obvious, making it difficult for students to progress. A paradigmatic example is the case of proofs by contradiction, which is a distinctive feature of classical logic. In some cases no direct proof exists, and the result can only be proved by contradiction: \(\varphi\) is proved by assuming \(\neg \varphi\) and showing that this leads to a contradiction. The feedback system should be able to identify such situations and suggest the appropriate rule applications.

\item \textbf {Breaking proofs into smaller sub-proofs:} Proofs in ND are incrementally built from smaller proofs. Dividing proofs into smaller steps reduces the cognitive load and simplifies reasoning. Therefore, the feedback system should encourage students to start with smaller proofs and incrementally build the main proof. 

\item \textbf{Indicating the distance to a solution:} Showing how many steps (rule applications) are needed to complete the proof helps students maintain focus and gain a clear sense of progress.

\item \textbf{Improvements in the proof:} Providing feedback about irrelevant steps taken or possible shortcuts that could make the proof clearer. It should also allow visualizing different ways to tackle the same problem.

\end{itemize}
Designing an algorithm that provides the basis for a feedback mechanism satisfying the above requirements is quite challenging. We aim to provide structured and clear information, as a tool for helping students to overcome the usual challenges of producing ND proofs.