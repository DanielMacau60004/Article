\section{Limitations}
The algorithm was developed for pedagogical purposes, so efficiency in proof generation was not our main focus. It can generate solutions for most exercises used in teaching environments but is more limited when searching for solutions in FOL proofs, as the solution space grows faster. Our algorithm is sound: if it finds a solution, it is definitely correct. This is guaranteed by the TG, which only generates valid transitions for each formula, and by the PG, which ensures that all goals in the proof are proved.  Regarding completeness, our algorithm is not complete because it can only find solutions up to a certain depth. Some proofs generate infinite graphs, so a solution may not be found. In most cases, this is not a problem, as we aim to find direct proofs. If a student’s proof deviates too much from the solution, it is not useful to provide feedback on that solution because the student is overcomplicating the problem. For example, if a teacher sees that a student is still working on a problem that can be solved in 10 steps, but the student’s current resolution already has 50 steps, even if a solution exists following the student’s approach, is it helpful to give feedback on it? Will the student really learn from that? Our algorithm stores multiple solutions for the same goal, which incurs memory costs since thousands of nodes may be stored in the final graph. However, this trade-off enables fast feedback generation, as the computational work has already been done.